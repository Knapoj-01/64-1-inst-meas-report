\documentclass[final,11pt]{article}
\usepackage[docversion = 0.0.1]{reportpackage}
%\setenumerate{label=\textbf{\arabic*}.}
\setlength{\baselineskip}{17pt}
\setmathrm{CMU Serif}[Scale=1.1]
\setlength{\parskip}{0.7em}
\renewcommand{\baselinestretch}{1.45}
\begin{document}
\section{บทนำ}
การวัดระดับ (Level Measurement) คือการระบุตำแหน่งของพื้นผิวภายในถัง เครื่องปฏิกรณ์
หรือภาชนะอื่นๆ โดยวัดระยะห่างแนวตั้ง (Vertical Distance) ระหว่างจุดอ้างอิงซึ่งโดยปกติคือฐานของภาชนะ กับพื้นผิว
ของของเหลว ของแข็ง หรือส่วนต่อประสานของของเหลวสองชนิด

การวัดระดับมีความสำคัญต่ออุตสาหกรรมเป็นอย่างมาก เพราะการทราบระดับของวัตถุดิบ และผลิตภัณฑ์ในกระบวนการผลิตต่างๆ
ทำให้สามารถจัดการระบบการผลิตได้อย่างมีแม่นยำ มีประสิทธิภาพ ช่วยเพิ่มความสามารถในการแข่งขันขององค์การ 
และที่สำคัญคือช่วยให้กระบวนการผลิตมีความปลอดภัย ซึ่งปัจจัยสำคัญทำให้ผู้ผลิต ได้รับไว้วางใจจากกลุ่มลูกค้า ผู้ลงทุน และประชาชนโดยรอบสถานที่ผลิต
โดยความสำคัญของการวัดระดับต่ออุตสาหกรรมในมิติต่างๆ พอจะสรุปได้ดังนี้ 
\begin{enumerate}
    \item \textbf{ประสิทธิภาพของกระบวนการผลิต} การทราบปริมาณที่แน่นอนจากการวัดระดับที่แม่นยำ
    ช่วยเพิ่มประสิทธิภาพของกระบวนการผลิต ผู้ผลิตสามารถจัดสรรทรัพยากรที่มีได้อย่างมีประสิทธิภาพ 
    และลดค่าใช้จ่ายในการจัดซื้อและบำรุงรักษาถังเก็บที่ไม่จำเป็น
    \item \textbf{ความปลอดภัย} การวัดระดับมีบทบาทอย่างมากในการรักษาความปลอดภัยในอุตสาหกรรม
    ความล้มเหลวในระบบวัดระดับ จนทำให้เกิดการบรรจุเกินจนล้น อาจนำไปสู่หายนะ ทำให้สารอันตรายเกิดการรั่วไหล
    สร้างความเสียหายต่อชีวิตและทรัพย์สิน รวมทั้งสิ่งแวดล้อมโดยรอบอย่างมหาศาลได้ 
    \item \textbf{มูลค่าของสินค้า} บ่อยครั้งมูลค่าของสินค้าที่เป็นของเหลว หรือของแข็งในถังเก็บ
    ขึ้นอยู่กับน้ำหนัก หรือปริมาตรของสินค้า ซึ่งคำนวณได้จากระดับของสินค้านั้นๆ 
    ความคลาดเคลื่อนในการวัดระดับเพียง $1/8$ นิ้ว ($\approx 3$ มิลลิเมตร) 
    จึงอาจส่งผลต่อมูลค่าของสินค้าได้อย่างมหาศาล โดยปกติเครื่องวัดที่ใช้วัดระดับในการซื้อขาย
    โอนกรรมสิทธิ์ในสินค้าตามกฏหมายจะมีความคลาดเคลื่อนในการวัดระดับน้อยกว่า  $1/16$ นิ้ว ($\approx 1$ มิลลิเมตร)
    และได้รับการอนุมัติจากหน่วยงานทางมาตรวิทยา
\end{enumerate}
\section{ความรู้พื้นฐานเกี่ยวกับการวัดระดับ และเครื่องวัดระดับ}
\subsection{ระบบวัดถัง}
\subsection{การคำนวณมวลและปริมาตรจากระดับ}
\subsection{ประเภทของเครื่องวัดระดับ}
\subsection{การเลือกใช้เครื่องวัดระดับ}

\newpage
\section{หลักการทำงานของเครื่องวัดระดับ และการประยุกต์ใช้ในอุตสาหกรรม}
\subsection{อุปกรณ์วัดระดับแบบลอย (Float Level Device)}
\subsubsection{หลักการทำงาน} 
อุปกรณ์วัดระดับแบบลอย เป็นเครื่องวัดระดับที่มีหลักการทำงานง่ายที่สุดแบบหนึ่ง
โดยเครื่องวัดระดับแบบลอยอย่างง่าย จะประกอบด้วยลูกลอยที่เป็นวัสดุลอยน้ำทรงกลม (Spherical) ทรงกระบอก (Cylindrical)
หรือเป็นทรงรีคล้ายแคปซูลยา (Oblong) ที่ถูกติดอยู่กับก้านกลไก เมื่อนำวัสดุลอยน้ำนี้ไปลอยน้ำ การเคลื่อนที่ของก้านกลไกจะเป็นไปโดยสอดคล้องกับระดับน้ำ
และสามารถเป็นตัวบอกระดับบนแผ่นเกจ (Guage Board) หรือกระตุ้นสวิตช์สำหรับส่งสัญญาณ หรือใช้ควบคุมการปิด-เปิดของเครื่องสูบน้ำ (Pump) ให้ทำงานได้โดยอัตโนมัติได้  

\subsubsection{รูปแบบของอุปกรณ์วัดระดับแบบลอย}
หลักการของเครื่องวัดระดับแบบลอย อาจนำไปสร้างเป็นเครื่องวัดในอุตสาหกรรมได้หลากหลายรูปแบบ โดยแต่ละรูปแบบก็จะเหมาะกับสภาพแวดล้อม 
และการใช้งานที่ต่างกันไป รูปแบบหลักๆ ของเครื่องวัดระดับแบบลอยสรุปได้ดังนี้
\begin{enumerate}
    \item \textbf{สวิตช์ลอยแบบเชื่อมต่อด้วยแม่เหล็ก (Magnetic Coupled Float Switch)} สวิตช์ลอยชนิดนี้ใช้การเชื่อมต่อทางแม่เหล็ก (Magnetic Coupling)
    ข้ามท่อปิดเพื่อแยกระหว่างของเหลวในกระบวนการและอุปกรณ์สวิตช์ของเครื่องวัด โดยมีโครงสร้างภายในแสดงได้ดังภาพที่ (*)
    เมื่อระดับของเหลวเพิ่มขึ้น ระดับของลูกลอยก็จะเพิ่มสูงขึ้น ปลายอีกด้านของก้านกลไกที่ติดอยู่กับลูกลอยซึ่งเป็นปลอกที่ทำมาจากวัสดุที่แม่เหล็กดูดติด (Attraction Sleeve) 
    จะถูกดันขึ้นจนเข้าใกล้แม่เหล็กที่อยู่อีกด้านของท่อ ทำให้แม่เหล็กนี้สามารถดูดติดวัสดุดังกล่าวได้ สวิตช์ปรอท (Mercury Switch) ที่อยู่อีกด้านของแขนแกว่งจะเอียงไปอีกด้านหนึ่งจนเกิดเปลี่ยนสถานะทางไฟฟ้า
    ในกรณีที่ต้องติดสวิตช์ลอยที่ผนังบ่อหรือถังเก็บ สวิตช์ลอยแบบเชื่อมต่อด้วยแม่เหล็กสามารถออกแบบได้อีกรูปแบบหนึ่งดังภาพที่ (*) โดยเมื่อระดับของเหลวเพิ่มสูงขึ้น
    แม่เหล็กพลิกกลับลงมากระตุ้นสวิตช์ให้เกิดการเปลี่ยนสถานะไป นอกจากนี้ยังอาจใช้หรีดสวิตช์ (Reed Switch) ที่เป็นหน้าสัมผัสขนาดเล็กบรรจุอยู่ในหลอดแก้ว 
    ช่วยลดขนาดและจำนวนส่วนเคลื่อนที่ ดังภาพที่ (*) และ (*) 
    \item \textbf{ลูกลอยและท่อนำ (Float and Guide Tube)} สวิตช์ลอยรูปแบบนี้ประกอบด้วยลูกลอยทรงกลมซึ่งถูกร้อยผ่านท่อนำที่ไม่ติดแม่เหล็ก
    แม่เหล็กรูปวงแหวนจะถูกฝังอยู่ในลูกลอย และหรีดสวิตช์จะถูกผนึกอยู่ในท่อนำดังภาพที่ (*) ในจุดที่ต้องการ เมื่อระดับของเหลวเพิ่มขึ้น ลูกลอยที่มีแม่เหล็กอยู่ภายใน
    จะลอยขึ้นตามท่อนำ โดยเมื่อลอยขึ้นถึงตำแหน่งที่ฝังหรีดสวิตช์ไว้ หรีดสวิตช์ก็จะเปลี่ยนสถานะไป
    \item \textbf{สวิตช์เอียง (Tilt Switch)} สวิตช์เอียงเป็นลูกลอยพลาสติกที่ภายในมีสวิตช์ปรอทบรรจุอยู่ และถูกแขวนอย่างอิสระโดยสายเคเบิล 
    ที่ระดับที่ต้องการ (ภาพที่ *) เมื่อระดับของเหลวเพิ่มขึ้นจนถึงระดับลูกลอย ลูกลอยจะเอียง และสวิตช์ปรอทจะเปลี่ยนสถานะ 
    สวิตช์ลอยลักษณะนี้จะใช้ได้กับการใช้งานภายใต้อุณหภูมิห้องและความดันบรรยากาศเท่านั้น
\end{enumerate}

\subsubsection{ข้อได้เปรียบและข้อจำกัด}
เครื่องวัดระดับแบบลอยมีหลักการทำงานที่ง่าย มีโครงสร้างไม่ซับซ้อนและส่วนประกอบไม่มาก ทำให้สามารถบำรุงรักษา ซ่อมแซมได้ง่าย และเชื่อถือได้เป็นอย่างมาก
นอกจากนี้เครื่องวัดระดับแบบลอยบางรูปแบบยังสามารถทำงานภายใต้อุณหภูมิและความดันที่สูง อย่างไรก็ตาม เครื่องวัดระดับแบบลอยเป็นอุปกรณ์เฉื่อยงาน (Passive) 
ที่ไม่มีระบบตรวจสอบตนเอง จึงมีความจำเป็นต้องทำการตรวจสอบอยู่เสมอ และเนื่องจากมีการสัมผัสกับของเหลวที่จะวัดโดยตรง ของเหลวที่หนืดอาจทำให้กลไกของลูกลอยเกิดติดขัดได้

\subsection{อุปกรณ์วัดระดับทางแสง (Optical Level Devices)}
อุปกรณ์วัดระดับทางแสงแบ่งตามหลักการทำงานได้เป็น 3 รูปแบบ ได้แก่การสะท้อน (Reflection), การส่งผ่าน (Transmission) และการหักเห (Refraction) 
โดยแต่ละรูปแบบก็จะมีข้อได้เปรียบและข้อจำกัดในการใช้งานที่แตกต่างกัน ในรายงานฉบับนี้จะกล่าวถึงรายละเอียดของอุปกรณ์วัดระดับทางแสงที่ใช้หลักการสะท้อน 
และการหักเหเท่าน้้น เนื่องจากอุปกรณ์วัดระดับทางแสงที่ใช้หลักการการส่งผ่าน มักใช้ตรวจจับความหนาของชั้นตะกอนของแข็งภายในของเหลว 
ซึ่งเกินขอบเขตของหัวข้อรายงานฉบับนี้

\subsubsection{หลักการทำงาน}
อุปกรณ์วัดระดับทางแสงแบบสะท้อนประกอบด้วยแหล่งกำเนิดแสง และเซนเซอร์ตรวจจับที่เป็นทรานซิสเตอร์ที่ไวต่อแสงหรือโฟโตทรานซิสเตอร์ (Phototransistor) 
บรรจุอยู่ในตัวถังเดียวกันดังภาพที่ (*) ลำแสงจากแหล่งกำเนิดแสงจะฉายไปยังของเหลวที่ทำการวัด และเมื่อระดับของของเหลวเพิ่มขึ้นจนถึงระดับหนึ่ง 
ลำแสงจะสะท้อนกลับไปยังเซนเซอร์รับแสง ทำให้โฟโตทรานซิสเตอร์นำกระแส ส่วนอุปกรณ์วัดระดับทางแสงประกอบด้วยแหล่งกำเนิดแสง 
และเซนเซอร์ตรวจจับ บรรจุอยู่ในตัวถังเดียวกันเช่นเดียวกับอุปกรณ์วัดระดับทางแสงแบบสะท้อน แต่สิ่งที่ต่างออกไปคือที่ปลายของหัววัดระดับจะถูกตัดเฉียง
เป็นมุม $45^\circ$ ดังภาพที่ (*) เมื่อเครื่องวัดถูกล้อมรอบด้วยอากาศ ลำแสงส่วนใหญ่จะสะท้อนกลับหมดภายในปริซึม ไปยังเซนเซอร์รับแสง ทำให้โฟโตทรานซิสเตอร์นำกระแส
เมื่อระดับของของเหลวที่ทำการวัดเพิ่มสูงขึ้นจนท่วมปริซึม แสงจากแหล่งกำเนิดแสงที่ตกกระทบปริซึมจะเกิดการหักเหเข้าสู่ของเหลวที่ทำการวัด เนื่องจากดัชนีหักเห
(Index of Refraction) ของของเหลวส่วนใหญ่จะมากกว่าอากาศ ทำให้โฟโตทรานซิสเตอร์หยุดนำกระแส

\subsubsection{ข้อได้เปรียบและข้อจำกัด}
อุปกรณ์วัดระดับทางแสงทั้งสองชนิดที่กล่าวมาก็มีข้อได้เปรียบและข้อจำกัดที่แตกต่างกันไป อุปกรณ์วัดระดับทางแสงแบบสะท้อนมีข้อได้เปรียบสำคัญคือ
ไม่มีการสัมผัสกับของเหลวที่ทำการวัด จึงสามารถใช้กับของเหลวที่กัดกร่อน เหนียวเหนอะหนะ และเคลือบติดได้ แต่ก็มีข้อเสียตรงที่ของเหลวที่ทำการวัดต้องทึบแสงพอสมควร
และต้องไม่มีไอ หรือความปั่นป่วนเหนือของเหลวที่ทำการวัด เพราะจะทำให้การวัดคลาดเคลื่อน ส่วนอุปกรณ์วัดระดับทางแสงแบบหักเหมีข้อดีสำคัญคือมีน้ำหนักเบา ขนาดเล็ก
และสามารถตรวจจับของเหลวปริมาณน้อยๆ ได้ แต่เนื่องจากหัววัดสัมผัสกับของเหลวโดยตรง จึงไม่สามารถใช้กับของเหลวที่กัดกร่อน เกาะติด หรือเคลือบติดได้ 
นอกจากนี้ยังอาจได้ผลบวกลวง (False Positive) หากมีหยดของของเหลวติดอยู่ที่ปลายหัววัด



\subsection{สวิตช์สภาพความนำ (Conductivity-type Level Switch)}
\subsubsection{หลักการทำงาน}
สภาพความนำ (Conductivity-$\sigma$) เป็นคุณสมบัติของวัสดุ ซึ่งถูกนิยามให้เป็นความนำ (Conductance- $G$) 
หรือส่วนกลับของความต้านทาน (Resistance-$R$) ของวัสดุหนึ่ง ที่มีพื้นที่หน้าตัด $\SI{1}{cm^2}$ และยาว $\SI{1}{cm}$ 
(หน่วยของสภาพความนำคือ $\SI{}{mS.cm^{-1}}$) สำหรับของเหลวหรือสารละลาย (Solution) 
สภาพความนำเป็นฟังก์ชันของจำนวนไอออนที่มีประจุที่อยู่ในของเหลวนั้นๆ ซึ่งโดยปกติแล้วจะมีมากกว่าในอากาศ ดังนั้น 
ถ้าวงจรไฟฟ้าถูกปิดโดยสารรอบๆ ปลายหัววัดหนึ่ง กระแสที่ไหลผ่านเมื่อหัววัดนี้ถูกจุ่มลงในสารละลาย จะมากกว่ากระแสที่ไหลเมื่อหัววัดนี้ถูกล้อมรอบด้วยอากาศ
อาศัยหลักการนี้ สวิตช์สภาพความนำจึงสามารถแยกระหว่างอากาศและสารละลาย หรือระหว่างสารละลายที่นำไฟฟ้า และสารละลายที่ไม่นำไฟฟ้าได้ 

พิจารณาสวิตช์สภาพความนำดังภาพที่ (*) เมื่อรอยต่อระหว่างของเหลวและอากาศเพิ่มขึ้นสูงถึงระดับหัววัด ของเหลวจะปิดวงจรทำให้กระแสไหลจากหัววัดหนึ่ง
ไปยังอีกหัววัดหนึงได้ โดยกระแสนี้จะถูกกำหนดให้มีค่าน้อยๆ อยู่ในระดับไมโครแอมป์ เพื่อป้องกันอันตรายจากไฟฟ้าดูดและการเกิดประกายไฟ 
และความไวของสวิตช์นี้จะถูกปรับให้เข้ากับสภาพความนำของของเหลวที่ทำการวัด หัววัดทั้งสองอาจมีความยาวไม่เท่ากันดังภาพที่ (*) หรืออาจมึการหน่วงเวลา 0 ถึง 20 วินาที 
เพื่อให้มีช่วงไร้การตอบสนอง (Dead zone) หรือช่วงสมดุล (Neutral zone) ซึ่งช่วยเพิ่มความเสถียรในกรณีที่มีการกวน หรือการกระฉอกภายในถัง 

\subsubsection{ข้อได้เปรียบและข้อจำกัด}
สวิตช์สภาพความนำมีโครงสร้างและหลักการทำงานที่ไม่ซับซ้อน ไม่มีส่วนเคลื่อนที่ที่สัมผัสกับของเหลว และสามารถนำไปใช้กับของแข็งที่ชื้นส่วนใหญ่ได้ 
ในด้านข้อจำกัด สวิตช์สภาพนำสามารถใช้ได้กับของเหลวที่นำไฟฟ้า ไม่กัดกร่อน และไม่เคลือบติดหัววัดเท่านั้น

\subsection{สวิตช์ระดับแบบสั่น}
\subsubsection{หลักการทำงาน}
สวิตช์ระดับแบบสั่นประกอบด้วยแหล่งกำเนิดความถี่เปียโซอิเล็กตริก (Piezoelectric Oscillator) ต่อกับหัววัดที่มีลักษณะเป็นแผ่นคล้ายใบมีด 
และตัวตรวจจับสัญญาณ ดังภาพที่ (*) โดยสัญญาณป้อนกลับที่ได้รับจากตัวตรวจจับจะถูกขยาย และนำไปขับผลึกเปียโซอิเล็กตริกที่ใช้กำเนิดสัญญาณอีกครั้ง
ในสภาวะปกติ หัววัดระดับนี้จะสั่นด้วยความถี่ธรรมชาติหนึ่ง แต่เมื่อระดับของวัสดุ (ของเหลวหรือของแข็งที่ทำการวัด) เพิ่มขึ้นจนสัมผัสกับหัววัด 
การสั่นของหัววัดจะถูกหน่วง (Damped) ทำให้ขนาดและความถี่ของการสั่นลดลง ตัวตรวจจับจึงสั่งการให้รีเลย์ทำการเปลี่ยนสถานะไป
\subsubsection{ข้อได้เปรียบและข้อจำกัด}
การวัดของสวิตช์ระดับแบบสั่นแทบไม่ได้รับผลกระทบจากการไหล (Flow), ความปั่นป่วน (Turbulence), ฟอง (Foam), การสั่น (Vibration)
การเคลือบ (Coating) และการเปลี่ยนแปลงคุณลักษณะของวัสดุ ทำให้เป็นเทคโนโลยีการวัดระดับที่สามารถใช้กับวัสดุได้หลากหลาย ไม่ว่าจะเป็นของเหลวชนิดต่างๆ,
ผงพลาสติค, นมผง, น้ำตาล, ข้าว, ธัญพืช, ฯลฯ และเชื่อถือได้มากเทคโนโลยีหนี่ง 
นอกจากนี้ยังไม่ต้องการการปรับเทียบ (Calibration) และมีความสามารถตรวจสอบการทำงานของตนเอง อุปกรณ์สมัยใหม่ที่ใช้ในอุตสาหกรรม 
สามารถตรวสอบและรายงานสถานะ และความผิดปกติต่างๆ ทั้งทางไฟฟ้าและทางกลได้อย่างสม่ำเสมอ อย่างไรก็ตาม 
สวิตช์ระดับแบบสั่นไม่สามารถใช้กับวัสดุที่มีความหนืดมากๆ เนื่องจากวัสดุอาจจับตัวระหว่างหัววัดทั้งสอง ทำให้เกิดความผิดพลาดได้หากไม่ได้รับการตรวจสอบอย่างสม่ำเสมอ 
\subsection{ความจุ}
\subsection{อัลตราโซนิค}
\subsection{เรดาร์แบบบังคับนำคลื่น}
\subsection{เรดาร์แบบไม่สัมผัส}
\section{บรรณานุกรม}
\end{document}