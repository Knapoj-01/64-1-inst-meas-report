\documentclass[final,11pt]{article}
\usepackage[docversion = 0.0.1]{reportpackage}
%\setenumerate{label=\textbf{\arabic*}.}
\setlength{\baselineskip}{17pt}
\setmathrm{CMU Serif}[Scale=1.1]
\setlength{\parskip}{0.7em}
\renewcommand{\baselinestretch}{1.45}
\begin{document}
\section{บทนำ}
การวัดระดับ (Level Measurement) คือการระบุตำแหน่งของพื้นผิวภายในถัง เครื่องปฏิกรณ์
หรือภาชนะอื่นๆ โดยวัดระยะห่างแนวตั้ง (Vertical Distance) ระหว่างจุดอ้างอิงซึ่งโดยปกติคือฐานของภาชนะ กับพื้นผิว
ของของเหลว ของแข็ง หรือส่วนต่อประสานของของเหลวสองชนิด

การวัดระดับมีความสำคัญต่ออุตสาหกรรมเป็นอย่างมาก เพราะการทราบระดับของวัตถุดิบ และผลิตภัณฑ์ในกระบวนการผลิตต่างๆ
ทำให้สามารถจัดการระบบการผลิตได้อย่างมีแม่นยำ มีประสิทธิภาพ ช่วยเพิ่มความสามารถในการแข่งขันขององค์การ 
และที่สำคัญคือช่วยให้กระบวนการผลิตมีความปลอดภัย ซึ่งปัจจัยสำคัญทำให้ผู้ผลิต ได้รับไว้วางใจจากกลุ่มลูกค้า ผู้ลงทุน และประชาชนโดยรอบสถานที่ผลิต
โดยความสำคัญของการวัดระดับต่ออุตสาหกรรมในมิติต่างๆ พอจะสรุปได้ดังนี้ 
\begin{enumerate}
    \item \textbf{ประสิทธิภาพของกระบวนการผลิต} การทราบปริมาณที่แน่นอนจากการวัดระดับที่แม่นยำ
    ช่วยเพิ่มประสิทธิภาพของกระบวนการผลิต ผู้ผลิตสามารถจัดสรรทรัพยากรที่มีได้อย่างมีประสิทธิภาพ 
    และลดค่าใช้จ่ายในการจัดซื้อและบำรุงรักษาถังเก็บที่ไม่จำเป็น
    \item \textbf{ความปลอดภัย} การวัดระดับมีบทบาทอย่างมากในการรักษาความปลอดภัยในอุตสาหกรรม
    ความล้มเหลวในระบบวัดระดับ จนทำให้เกิดการบรรจุเกินจนล้น อาจนำไปสู่หายนะ ทำให้สารอันตรายเกิดการรั่วไหล
    สร้างความเสียหายต่อชีวิตและทรัพย์สิน รวมทั้งสิ่งแวดล้อมโดยรอบอย่างมหาศาลได้ 
    \item \textbf{มูลค่าของสินค้า} บ่อยครั้งมูลค่าของสินค้าที่เป็นของเหลว หรือของแข็งในถังเก็บ
    ขึ้นอยู่กับน้ำหนัก หรือปริมาตรของสินค้า ซึ่งคำนวณได้จากระดับของสินค้านั้นๆ 
    ความคลาดเคลื่อนในการวัดระดับเพียง $1/8$ นิ้ว ($\approx 3$ มิลลิเมตร) 
    จึงอาจส่งผลต่อมูลค่าของสินค้าได้อย่างมหาศาล โดยปกติเครื่องวัดที่ใช้วัดระดับในการซื้อขาย
    โอนกรรมสิทธิ์ในสินค้าตามกฏหมายจะมีความคลาดเคลื่อนในการวัดระดับน้อยกว่า  $1/16$ นิ้ว ($\approx 1$ มิลลิเมตร)
    และได้รับการอนุมัติจากหน่วยงานทางมาตรวิทยา
\end{enumerate}
\section{ความรู้พื้นฐานเกี่ยวกับการวัดระดับ และเครื่องวัดระดับ}
\subsection{ระบบวัดถัง}
\subsection{การคำนวณมวลและปริมาตรจากระดับ}
\subsection{ประเภทของเครื่องวัดระดับ}
\subsection{การเลือกใช้เครื่องวัดระดับ}
\section{หลักการทำงานของเครื่องวัดระดับ และการประยุกต์ใช้ในอุตสาหกรรม}
\subsection{เครื่องวัดระดับแบบจุด}
\subsubsection{ลูกลอย}
\subsubsection{สวิตช์ทางแสง}
\subsubsection{ความนำ}
\subsubsection{ส้อมเสียง}
\subsection{เครื่องวัดระดับแบบต่อเนื่อง}
\subsubsection{ความจุ}
\subsubsection{อัลตราโซนิค}
\subsubsection{เรดาร์แบบบังคับนำคลื่น}
\subsubsection{เรดาร์แบบไม่สัมผัส}
\section{บรรณานุกรม}
\end{document}